%!TEX root = ../dokumentation.tex

\chapter*{Abkürzungsverzeichnis}
%nur verwendete Akronyme werden letztlich im Dokument angezeigt
\begin{acronym}[YTMMM]
\setlength{\itemsep}{-\parsep}

\acro{IDE}{Integrierte Entwicklungsumgebung}
\acro{RCP}{Rich Client Platform}
\acro{API}{Application Programming Interface}
\acro{WYSIWYG}{What You See Is What You Get}
\acro{OSGi}{Open Services Gateway initiative}
\acro{SWT}{Standart Widget Toolkit}
\acro{DBWB}{DockBridge Workbench}
\acro{GUI}{Graphical User Interface}
\acro{AFP}{Apple Filing Protocol}
\acro{PDF}{Portable Document Format}
\acro{XML}{Extensible Markup Language}
\acro{HTML}{Hypertext Markup Language}
\acro{XHTML}{Extensible Hypertext Markup Language}
\acro{JRE}{Java Runtime Environment}
\acro{MVC}{Model View Controller}
\acro{SAX}{Simple API for XML}
\acro{DOM}{Document Object Model}
\acro{ASE}{Apache Software Foundation}
\acro{JaxB}{Java Architecture for XML Binding}
\end{acronym}
