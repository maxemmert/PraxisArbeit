%!TEX root = ../dokumentation.tex

%
% vorher in Konsole folgendes aufrufen: 
%	makeglossaries makeglossaries dokumentation.acn && makeglossaries dokumentation.glo
%

%
% Glossareintraege --> referenz, name, beschreibung
% Aufruf mit \gls{...}
%
\newglossaryentry{Glossareintrag}{name={Glossareintrag},plural={Glossareinträge},description={Ein Glossar beschreibt verschiedenste Dinge in kurzen Worten}}

\newglossaryentry{Mockup}{name={Mockup},plural={Mockups},description={Ein Mockup ist ein maßstäbliches Modell, das oft zu Präsentationszwecken oder zur Veranschaulichung genutzt wird}}

\newglossaryentry{Toolbar}{name={Toolbar},plural={Toolbars},description={Eine Toolbar (Statusleiste) ist eine waagerechte oder senkrechte Leiste mit kleinen, häufig bebilderten Schaltflächen, die als erweiternde Elemente der Menüführung von Programmen mit grafischer Benutzeroberfläche dienen}}