%!TEX root = ../dokumentation.tex

%
% vorher in Konsole folgendes aufrufen: 
%	makeglossaries makeglossaries dokumentation.acn && makeglossaries dokumentation.glo
%

%
% Glossareintraege --> referenz, name, beschreibung
% Aufruf mit \gls{...}
%
\newglossaryentry{Glossareintrag}{name={Glossareintrag},plural={Glossareinträge},description={Ein Glossar beschreibt verschiedenste Dinge in kurzen Worten}}

\newglossaryentry{Mockup}{name={Mockup},plural={Mockups},description={Ein Mockup ist ein maßstäbliches Modell, das oft zu Präsentationszwecken oder zur Veranschaulichung genutzt wird}}

\newglossaryentry{Toolbar}{name={Toolbar},plural={Toolbars},description={Eine Toolbar (Statusleiste) ist eine waagerechte oder senkrechte Leiste mit kleinen, häufig bebilderten Schaltflächen, die als erweiternde Elemente der Menüführung von Programmen mit grafischer Benutzeroberfläche dienen}}

\newglossaryentry{String}{name={String},plural={Strings},description={Ein String ist eine Zeichenkette bzw. eine Folge von Zeichen aus einem definierten Zeichensatz}}

\newglossaryentry{Framework}{name={Framework},plural={Frameworks},description={Programmiergerüst, das in der Softwaretechnik im Rahmen der objektorientierten Softwareentwicklung sowie bei komponentenbasierten Entwicklungsansätzen verwendet wird}}

\newglossaryentry{Logging}{name={Logging},plural={Loggings},description={(Automatische) Speicherung von Prozessen und Datenänderungen}}

\newglossaryentry{Plug-in}{name={Plug-in},plural={Plug-ins},description={Erweiterungsmodul, das von einer Software während ihrer Laufzeit erkannt und angeschlossen werden kann. Mit Plug-ins werden Applikationen um zusätzliche Funktionalitäten erweitert}}

\newglossaryentry{Repository}{name={Repository},plural={Repositories},description={Verwaltetes Verzeichnis zur Speicherung und Beschreibung von digitalen Objekten (oft genutzt im Zusammenhang mit Versionsverwaltung)}}

\newglossaryentry{Versionsverwaltung}{name={Versionsverwaltung},plural={Versionsverwaltungen},description={System, das zur Erfassung von Änderungen an Dokumenten oder Dateien verwendet wird. Alle Versionen werden in einem Archiv oder Repository abgelegt, um ungewollte Datenverluste zu vermeiden}}




