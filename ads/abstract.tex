%!TEX root = ../dokumentation.tex

\pagestyle{empty}

\renewcommand{\abstractname}{Zusammenfassung}
\begin{abstract}
\textbf{Thema der Arbeit}{
\newline
Einbindung von Plug-ins im Eclipse \ac{RCP}-Umfeld, sowie deren grafische Einbettung in eine bestehende \ac{RCP}-Anwendung auf Basis von OSGi. 
}
\newline
\textbf{Stichworte}{
\newline
OSGi, Eclipse RCP, GUI, Filterprofil, Plug-in
}
\newline
\textbf{Kurzzusammenfassung}{
\newline
In dieser Arbeit wird untersucht, inwiefern es möglich ist, Plug-ins in eine bestehende Eclipse \ac{RCP}-Applikation zu integrieren. Dabei steht vor allem die grafische Einbindung in das bestehende Softwarekonstrukt im Vordergrund. Die Konzeption und Umsetzung eines Plug-ins gehört ebenfalls zu den Themen. Es werden Grundlagen der Eclipse \ac{RCP}-Entwicklung und für diese notwendige Technologien untersucht. Ein weiterer Fokus dieser Arbeit liegt auf dem Thema OSGi.    

}




\end{abstract}





\renewcommand{\abstractname}{Summary}
\begin{abstract}



\end{abstract}
