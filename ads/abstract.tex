%!TEX root = ../dokumentation.tex

\pagestyle{empty}

\renewcommand{\abstractname}{Zusammenfassung}
\begin{abstract}
\textbf{Thema der Arbeit}{
\newline
Einbindung von Plug-ins im Eclipse \ac{RCP}-Umfeld, sowie deren grafische Einbettung in eine bestehende \ac{RCP}-Anwendung auf Basis von \ac{OSGi}. 
}
\newline
\textbf{Stichworte}{
\newline
OSGi, Eclipse RCP, GUI, Filterprofil, Plug-in
}
\newline
\textbf{Kurzzusammenfassung}{
\newline
In dieser Arbeit wird untersucht, inwiefern es möglich ist, Plug-ins in eine bestehende Eclipse \ac{RCP}-Applikation zu integrieren. Dabei steht vor allem die grafische Einbindung in das bestehende Softwarekonstrukt im Vordergrund. Die Konzeption und Umsetzung eines Plug-ins im \ac{RCP}-Umfeld gehört ebenfalls zum Kern dieser Arbeit. Neben den Grundlagen der Eclipse \ac{RCP} Entwicklung werden verwandte Themen, wie beispielsweise \ac{OSGi} behandelt.

}




\end{abstract}





\renewcommand{\abstractname}{Summary}
\begin{abstract}
\textbf{Thema der Arbeit}{
\newline
Integration of plug-ins in an existing Eclipse \ac{RCP} environment, as well as their graphical embedding in an \ac{RCP} application based on \ac{OSGi}.
}
\newline
\textbf{Stichworte}{
\newline
OSGi, Eclipse RCP, GUI, Filterprofil, Plug-in
}
\newline
\textbf{Kurzzusammenfassung}{
\newline
In this work it is studied how to extent it is possible to integrate plug-in into an existing Eclipse \ac{RCP} application. In particular, the graphic integration into the existing software construct is in the foreground. The design and implementation of a plugin in the \ac{RCP} environment is also part of the core of this work. Besides the basics of Eclipse \ac{RCP} development related issues, such as \ac{OSGi} are treated.

}




\end{abstract}
