%!TEX root = ../dokumentation.tex

\chapter{Einleitung}
\label{cha:Einleitung}

\section{Motivation}{
Noch vor einigen Jahren war es nicht unüblich, Anwendungen über die Kommandozeile zu steuern. Die funktionalen Aspekte einer Anwendung hatten eine höhere Priorität als ihre Nutzbarkeit. Um komplizierte Applikationen über die Kommandozeile zu bedienen ist jedoch oft Expertenwissen notwendig. Deshalb spielt die Entwicklung von grafischen Oberflächen in der heutigen Anwendungsentwicklung eine immer größere Rolle. Es reicht oft nicht aus, dem Benutzer nur eine grafische Möglichkeit zur Steuerung einer Applikation zu geben. Im Rahmen der Software-Ergonomie spielt die Benutzerführung eine tragende Rolle. So auch bei der Anwendung Workbench for Mill Plus, eine Anwendung der Compart AG, die es dem Anwender ermöglicht bequem Prozesse zur Modifikation und Konvertierung von Dokumenten zu steuern. Der Anwendung Workbench for Mill Plus liegt die Programmiersprache Java zu Grunde. Durch die hohe Verbreitung von Java entstehen immer komplexere Anwendungen. Um die Komplexität solcher Anwendungen handhaben zu können bedient man sich oft Mitteln zur Modularisierung. Hierfür hat Java jedoch keine eigene Sprachunterstützung. \ac{OSGi} bietet die Möglichkeit, monolithischen Anwendungen, die den heutigen dynamischen Anforderungen nicht gerecht werden, entgegenzuwirken.
}
\section{Zielsetzung}{
In der vorausgegangenen Praxisphase wurde vom Studenten ein Programm entwickelt, das es ermöglicht Filterprofile anhand eines gegebenen XML-Schemas zu aktualisieren. Diese Funktionalität soll nun in die Workbench for Mill Plus integriert werden. Dies würde Kunden bei der Auslieferung eines neuen XML-Schemas die Möglichkeit geben, die Konfiguration ihrer alten Profile beizubehalten beziehungsweise ein neues Filterprofil zu erstellen, das die Konfiguration des alten Filterprofils enthält, jedoch zum neuen XML-Schema valide ist. Darüber hinaus soll es dem Anwender möglich sein, Profile in die Workbench for Mill Plus zu importieren. Wird beim Import festgestellt, dass das Filterprofil nicht dem gegebenen XML-Schema entspricht soll dies dem Anwender kenntlich gemacht werden. 
}
\section{Aufbau der Arbeit}{
Das Kapitel \nameref{cha:Grundlagen} soll dem Leser einen Überblick über das Thema verschaffen. Es werden die Motivation und die Zielsetzung dieser Arbeit erläutert. Darüber hinaus werden Begriffe und Systeme erklärt, die für das Verständnis und Nachvollziehbarkeit der Arbeit grundlegend sind. Im \nameref{cha:Hauptteil} wird die Vorgehensweise zur Problemlösung detailliert beschrieben.
}
